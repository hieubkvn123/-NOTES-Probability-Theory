\subsection{Overview of measure theory}
\subsubsection{Algebra and $\sigma$-algebra}
\begin{definition}[Algebra]
    Let $X$ be a set and $\mathcal{A}$ be a collection of subsets of $X$. Then, we say that $\mathcal{A}$ is an algebra if it satisfies:
    \begin{itemize}
        \item \textbf{Closure under complement} : If $E \in \mathcal{A} \implies E^c \in \mathcal{A}$.
        \item \textbf{Closure under finite union} : For all finite collection $\{E_n\}_{n=1}^N\subset \mathcal{A} \implies \bigcup_{n=1}^N E_n \in \mathcal{A}$.
    \end{itemize}
\end{definition}

\begin{definition}[$\sigma$-algebra]
    Let $X$ be a set and $\mathcal{A}$ be a collection of subsets of $X$. Then, we say that $\mathcal{A}$ is a $\sigma$-algebra if it is:
    \begin{itemize}
        \item \textbf{Closure under complement} : If $E \in \mathcal{A} \implies E^c \in \mathcal{A}$.
        \item \textbf{Closure under countable union} : For all countable collection $\{E_n\}_{n=1}^\infty\subset \mathcal{A} \implies \bigcup_{n=1}^\infty E_n \in \mathcal{A}$.
    \end{itemize}
\end{definition}

\begin{definition}[Borel-$\sigma$-algebra]
    Let $\Sigma$ be the set of all the $\sigma$-algebras generated by open intervals in $\mathbb{R}$. Then, the Borel-$\sigma$-algebra is the smallest $\sigma$-algebra containing the open intervals:
    \begin{align*}
        \mathcal{B} = \bigcap_{\mathcal{A}\in\Sigma} \mathcal{A}
    \end{align*}
\end{definition}

\begin{proposition}{Disjoint union in algebra}{disjoint_union_in_algebra}
    Let $\mathcal{A}$ be an algebra and let $\{E_n\}_{n=1}^\infty$ be a countable collection of subsets in $\mathcal{A}$. Then, there exists a countable disjoint subsets $\{F_n\}_{n=1}^\infty$ such that:
    \begin{align*}
        \bigcup_{n=1}^\infty E_n = \bigcup_{n=1}^\infty F_n
    \end{align*}
\end{proposition}

\begin{proof*}[Proposition \ref{prop:disjoint_union_in_algebra}]
    Let $G_m = \bigcup_{n=1}^m E_n$, we have $G_1 \subset G_2 \subset G_3 \subset \dots \subset G_N$. It is easy to see that $\bigcup_{n=1}^N G_n = \bigcup_{n=1}^N E_n$. Now, define the collection $\{F_n\}_{n=1}^\infty$ as followed:
    \begin{align*}
        F_n = 
        \begin{cases}
            G_1 & \text{When } n = 1
            \\ \\
            G_{n} \setminus G_{n-1} & \text{When } n \ge 2
        \end{cases}
    \end{align*}

    \noindent Hence, we have $\bigcup_{n=1}^N F_n = \bigcup_{n=1}^N G_n \implies \bigcup_{n=1}^NF_n = \bigcup_{n=1}^N E_n$.
\end{proof*}



\subsubsection{Measurable spaces}
\begin{definition}[Measurable space]
    Let $E$ be a set and $\mathcal{E}$ be a $\sigma$-algebra over $E$. Then, the pair $(E, \mathcal{E})$ is called a \textbf{measurable space}. The elements in $\mathcal{E}$ are called \textbf{measurable sets}. When $E$ is a topological space and $\mathcal{E}$ is the Borel-$\sigma$-algebra on $E$, then the elements in $\mathcal{E}$ are also called \textbf{Borel sets}.
\end{definition}

\begin{definition}[Product of measurable spaces]
    Let $(E, \mathcal{E})$ and $(F, \mathcal{F})$ be measurable spaces. For $A\subset E, B\subset F$, we denote the product of $A, B$, denoted $A\times B$, as the set of all pairs $(x, y)$ such that $x \in A, y\in B$. The set $A\times B$ is then called a \textbf{measurable rectangle}. The measurable space $(E\times F, \mathcal{E} \otimes \mathcal{F})$ is called the product of measurable spaces $(E, \mathcal{E})$ and $(F, \mathcal{F})$ where $\mathcal{E} \otimes \mathcal{F}$ is a $\sigma$-algebra over $E\times F$:
    \begin{align*}
        \mathcal{E} \otimes \mathcal{F} = \Big\{ A \times B: A \in \mathcal{E}, B \in \mathcal{F} \Big\}
    \end{align*}
\end{definition}

\subsubsection{Measures \& Measure space}
\begin{definition}[Measure \& Measure space]
    Let $(E, \mathcal{E})$ be a measurable space. A measure is a mapping $\mu:\mathcal{E}\to[0, \infty]$ (Including infinity) such that:
    \begin{itemize}
        \item \textbf{Empty set has zero measure} : $\mu(\emptyset) = 0$.
        \item \textbf{Countable (disjoint) additivity} : For a collection of disjoint measurable sets $\{E_n\}_{n=1}^\infty$, we have
        \begin{align*}
            \mu\Bigg( \bigcup_{n=1}^\infty E_n \Bigg) = \sum_{n=1}^\infty \mu(E_n)
        \end{align*}
    \end{itemize}

    \noindent The triplet $(E, \mathcal{E}, \mu)$ is called the $\textbf{Measure space}$ and $\mu$ is called a measure on the measurable space $(E, \mathcal{E})$.
\end{definition}

\noindent\newline \textbf{Remark} : Note that \textbf{translation invariance} is not included because this property is specific to Lebesgue measure only. A general measure need not to have translation invariance.  

\noindent\newline \textbf{Examples} : Here are some of the most common examples of measures
\begin{itemize}
    \item \textbf{Dirac measures $\delta_x$} : Let $(E, \mathcal{E})$ be a measurable space and let $x\in E$ be a fixed point. For all $A\in\mathcal{E}$, defined:
    \begin{align*}
        \delta_x(A) = \begin{cases}
            1 &\text{if } x \in A 
            \\ \\
            0 &\text{if } x \notin A
        \end{cases}
    \end{align*}

    \noindent Then, $\delta_x$ is a measure on $(E, \mathcal{E})$ and it is called the \textbf{Dirac measure sitting at $x$}.

    \item \textbf{Counting measures} : Let $(E, \mathcal{E})$ be a measurable space and $D\subset E$ be countable. For each $A\in\mathcal{E}$, $\nu_D(A)$ is the number of points in $A\cap D$:
    \begin{align*}
        \nu_D(A) = \sum_{x\in D}\delta_x(A), \ \ A \in \mathcal{A}
    \end{align*}

    \item \textbf{Discrete measures} : Let $(E, \mathcal{E})$ be a measurable space and $D\subset E$ be countable. For each $x\in D$, define $m:D\to (0, \infty)$ be a function that assigns a positive number to $x$. Define:
    \begin{align*}
        \nu_D^m (A) = \sum_{x\in D}m(x)\delta_x(A), \ \ A \in \mathcal{A}
    \end{align*}

    \noindent Then, $\nu_D^m$ is called a \textbf{discrete measure} on $(E, \mathcal{E})$. We can understand $m(x)$ as a mass attached to each point $x\in D$.
\end{itemize}

\begin{proposition}{Properties of measures}{properties_of_measures}
    Let $\mu$ be a measure on a measurable space $(X, \mathcal{A})$. Then, the following properties hold for all measurable sets $A, B$ and a countable collection (not necessarily disjoint) of measurable sets $\{E_n\}_{n=1}^\infty$.
    \begin{itemize}
        \item \textbf{Monotonicity} : $A \subseteq B \implies \mu(A) \le \mu(B)$.
        \item \textbf{Countable sub-additivity} : $\mu\Big( \bigcup_{n=1}^\infty E_n \Big) \le \sum_{n=1}^\infty \mu(E_n)$.
        \item \textbf{Continuity}:
        \begin{itemize}
            \item \textbf{Continuity from below} : $E_n \uparrow E\implies \mu(E_n) \uparrow \mu(E)$.
            \item \textbf{Continuity from above} : $E_n \downarrow E\implies \mu(E_n) \downarrow \mu(E)$.
        \end{itemize}
    \end{itemize}
\end{proposition}
\begin{proof*}[Proposition \ref{prop:properties_of_measures}]
    We prove each property one by one:
    \begin{subproof}{\newline Monotonicity}
        If $A \subseteq B$, we have:
        \begin{align*}
            \mu(B) &= \mu((B\setminus A) \cup A) \\
            &= \mu(B \setminus A) + \mu(A) \ \ \ (\text{Countable (disjoint) additivity}) \\
            &\ge \mu(A)
        \end{align*}
    \end{subproof}

    \begin{subproof}{\newline Countable sub-additivity}
        For two measurable sets $A, B$, we have:
        \begin{align*}
            \mu(A\cup B) &= \mu((A\cup B)\setminus A) + \mu(A) \\
            &= \mu(B\setminus A) + \mu(A) \\
            &\le \mu(B) + \mu(A)
        \end{align*}

        \noindent Hence, extend the argument inductively, for a countable collection $\{E_n\}_{n=1}^\infty$, we have:
        \begin{align*}
            \mu\Big( \bigcup_{n=1}^\infty E_n \Big) \le \sum_{n=1}^\infty \mu(E_n)
        \end{align*}
    \end{subproof}

    \begin{subproof}{\newline Continuity}
        \textbf{$\bf (i)$ Continuity from below} : Let $\{E_n\}_{n=1}^\infty$ be an increasing collection of measurable sets such that $E_1\subseteq E_2 \subseteq \dots$ and $\bigcup_{n=1}^\infty E_n = E$. Construct a countable collection of disjoint measurable sets $\{F_n\}_{n=1}^\infty$ such that:
        \begin{align*}
            \begin{cases}
                F_1 &= E_1
                \\ \\
                F_n &= E_n \setminus E_{n-1}, \ n \ge 2
            \end{cases}
        \end{align*}

        \noindent Apparently $\{F_n\}_{n=1}^\infty$ is a disjoint collection and we have $E_n = \bigcup_{k=1}^n F_k$. Hence, we have:
        \begin{align*}
            \lim_{n\to\infty} \mu(E_n) 
                &= \lim_{n\to\infty} \mu\Bigg(\bigcup_{k=1}^n F_k\Bigg) \\
                &= \lim_{n\to\infty} \sum_{k=1}^n \mu(F_k) = \sum_{k=1}^\infty \mu(F_k) \\
                &= \mu\Bigg( \bigcup_{k=1}^\infty F_k \Bigg) = \mu\Bigg( \bigcup_{n=1}^\infty E_n \Bigg) = \mu(E) \\
        \end{align*}

        \noindent \textbf{$\bf (ii)$ Continuity from above} : Let $\{E_n\}_{n=1}^\infty$ be an decreasing collection of measurable sets such that $E_1\supseteq E_2 \supseteq \dots$ and $\bigcap_{n=1}^\infty E_n = E$. We have:
        \begin{align*}
            \mu(E) &= \mu\Bigg( \bigcap_{n=1}^\infty E_n \Bigg) \\
            &= \mu\Bigg( \Bigg[ \bigcup_{n=1}^\infty E_n^c\Bigg]^c \Bigg) = \mu\Bigg( X \setminus \Bigg[ \bigcup_{n=1}^\infty E_n^c\Bigg] \Bigg) \\
            &= \mu(X) - \mu\Bigg( \bigcup_{n=1}^\infty E_n^c \Bigg) \ \ \ \text{(By monotonicity)} \\
            &= \mu(X) - \lim_{n\to\infty} \mu(E_n^c) \ \ \ \text{(As proven in $\bf (i)$)} \\
            &= \lim_{n\to\infty} \mu(X\setminus E_n^c) = \lim_{n\to\infty} \mu(E_n)
        \end{align*}
    \end{subproof}
\end{proof*}




\subsubsection{Lebesgue Measure}
\noindent \textbf{Overview} : The definition of Lebesgue measure stems from the need to construct a more general notion of integral (the Lebesgue integral) because the simple notion of Riemann integral is incomplete. For example, $L^1_R([0,1])$ (space of absolutely Riemann-integrable functions) is not a Banach space.

\noindent\newline The construction of the Lebesgue integral over $\mathbb{R}$ requires a notion of "measure" on subsets of $\mathbb{R}$, which, ideally satisfies the following conditions:
\begin{itemize}
    \item $\mu:\mathcal{P}(\mathbb{R}) \to [0, \infty)$ where $\mathcal{P}(\mathbb{R})$ denotes the power set of $\mathbb{R}$.
    \item $\mu$ \textbf{extends the measure of interval length} $l$. Meaning, if $I\subset \mathbb{R}$ is an interval, $\mu(I)=l(I)$.
    \item \textbf{Countable additivity} : Let $\{E_n\}_{n=1}^\infty\subset X$ be a collection of disjoint subsets of $\mathbb{R}$, then $\mu(\bigcup_{n=1}^\infty E_n) = \sum_{n=1}^\infty \mu(E_n)$.
    \item \textbf{Translation invariance} : For $E \subset \mathbb{R}, x\in \mathbb{R}$, we have $\mu(E+x)=\mu(E)$.
\end{itemize}

\noindent However, it is widely known that the construction of such measure is not possible because of the existence of non-measurable sets (Vitali sets \cite{wiki:vitaliset}).

\begin{definition}[Lebesgue outer measure]
    Let $E\subset \mathbb{R}$. The Lebesgue outer measure (or simply "outer measure") is a mapping from the power set of $\mathbb{R}$ to $[0, \infty)$ such that:
    \begin{align*}
        \mu^*(E) = \inf\Bigg\{ \sum_{n=1}^\infty l(I_n) : I_n \text{ are open intervals}; E \subseteq \bigcup_{n=1}^\infty I_n \Bigg\}
    \end{align*}

    \noindent Where $l$ denotes interval length. Without proving, we will just acknowledge the fact that the Lebesgue outer measure satisfies the second and the fourth conditions. However, \textbf{the outer measure is countably sub-additive rather than countably additive}. To account for this, we look at the definition of the Caratheodory criterion below.
\end{definition}

\begin{definition}[Caratheodory criterion - Lebesgue measurable sets]
    Let $E \subseteq \mathbb{R}$. The set $E$ is called \textbf{Lebgesgue measurable} if for all $A\subseteq \mathbb{R}$, we have:
    \begin{align*}
        \mu^*(A) = \mu^*(A \cap E) + \mu^*(A\cap E^c)
    \end{align*}
    \noindent The above condition is called the \textbf{Caratheodory criterion}. We denote the set of Lebesgue measurable subsets as $\mathcal{M}$:
    \begin{align*}
        \mathcal{M} = \Bigg\{ E \subseteq \mathbb{R} : \forall A \subseteq \mathbb{R}, \mu^*(A) = \mu^*(A \cap E) + \mu^*(A\cap E^c) \Bigg\}
    \end{align*}
\end{definition}

\noindent \newline \textbf{Remark} : Note that by countable sub-additivity, we will always have $\mu^*(A) \ge \mu^*(A\cap E)$

\begin{definition}[Lebesgue measure]
    The Lebesgue measure (denoted $\mu$) is simply the Lebesgue outer measure $\mu^*$ restricted to the set of Lebesgue measurable sets $\mathcal{M}$:
    \begin{align*}
        \mu : \mathcal{M} \to [0, \infty); \ \ \mu := \mu^*\Big|_{\mathcal{M}}
    \end{align*}
\end{definition}

\begin{proposition}{Measure of intersection with measurable collection}{intersection_with_measurable_collection}
    Let $A\subseteq \mathbb{R}$ and let $\{E_n\}_{n=1}^N$ be a finite disjoint collection of Lebesgue measurable sets. Then, we have:
    \begin{align*}
        \mu^*\Bigg( A \cap \Bigg[ \bigcup_{n=1}^N E_n \Bigg] \Bigg) = \sum_{n=1}^N \mu^*(A\cap E_n)
    \end{align*}
\end{proposition}

\begin{proof*}[Proposition \ref{prop:intersection_with_measurable_collection}]
    We will prove this by induction. For $N=1$, both sides are identical. For the inductive step, suppose that the above proposition is true for $N=m$. We have to prove that it is true for $N=m+1$.

    \noindent \newline Since $E_{m+1}$ is measurable, using the Caratheodory criterion, we have:
    \begin{align*}
        \mu^*\Bigg( A \cap \Bigg[ \bigcup_{n=1}^{m+1} E_n \Bigg] \Bigg) 
            &= \mu^*\Bigg( A \cap \Bigg[ \bigcup_{n=1}^{m+1} E_n \Bigg] \cap E_{m+1} \Bigg)
            + \mu^*\Bigg( A \cap \Bigg[ \bigcup_{n=1}^{m+1} E_n \Bigg] \cap E_{m+1}^c \Bigg) \\
            &= \mu^*(A\cap E_{m+1}) + \mu^*\Bigg( A \cap \Bigg[ \bigcup_{n=1}^{m+1} E_n \Bigg] \cap E_{m+1}^c \Bigg)
    \end{align*}

    \noindent \newline Since $E_{n}$ is disjoint for all $1 \le n \le m+1$. We have:
    \begin{align*}
        \bigcup_{n=1}^{m} E_n \subset E_{m+1}^c \implies \Bigg[ \bigcup_{n=1}^{m+1} E_n \Bigg] \cap E_{m+1}^c = \bigcup_{n=1}^m E_n
    \end{align*}

    \noindent \newline Finally, we have
    \begin{align*}
        \mu^*\Bigg( A \cap \Bigg[ \bigcup_{n=1}^{m+1} E_n \Bigg] \Bigg) 
            &= \mu^*(A\cap E_{m+1}) + \mu^*\Bigg( A \cap \Bigg[ \bigcup_{n=1}^{m} E_n \Bigg] \Bigg) \\
            &= \mu^*(A\cap E_{m+1}) + \sum_{n=1}^m \mu^*(A\cap E_n) \\
            &= \sum_{n=1}^{m+1} \mu^*(A\cap E_n)
    \end{align*}
\end{proof*}

\begin{proposition}{$\mathcal{M}$ is $\sigma$-algebra}{m_is_sigma_algebra}
    The set of Lebesgue measurable subsets $\mathcal{M}$ is a $\sigma$-algebra.
\end{proposition}
\begin{proof*}[Proposition \ref{prop:m_is_sigma_algebra}]
    We first prove that $\mathcal{M}$ is an algebra. Then, for all countable collection of Lebesgue measurable sets $\{E_n\}_{n=1}^\infty$ such that $E = \bigcup_{n=1}^\infty E_n$, $\mu^*(A)\ge\mu^*(A\cap E) + \mu^(A\cap E^c)$.
    
    \begin{subproof}{\newline Claim 1 : $\mathcal{M}$ is an algebra}
        We have to prove that $\mathcal{M}$ is both closed under complement and finite union.
        \begin{itemize}
            \item \textbf{Closure under complement} : Trivial due to the symmetry of the Caratheodory criterion.
            \item \textbf{Closure under finite union} : Let $E_1, E_2$ be two Lebesgue measurable sets. We have:
        
            \begin{align*}
                \mu^*(A\cap(E_1\cup E_2)) &= \mu^*((A \cap E_1) \cup (A\cap E_2)) \\
                &=   \mu^*((A \cap E_1) \cup (A\cap E_2 \cap E_1^c)) \\
                &\le \mu^*(A\cap E_1) + \mu^*(A\cap E_2 \cap E_1^c) \ \ \ \text{(Countable sub-additivity)} \\
                &=   \mu^*(A) - \mu^*(A\cap E_1^c) + \mu^*(A\cap E_2 \cap E_1^c) \\
                &=   \mu^*(A) - \Big[ \mu^*(A\cap E_1^c) - \mu^*([A\cap E_1^c] \cap E_2) \Big] \\
                &=   \mu^*(A) - \mu^*(A\cap E_1^c \cap E_2^c) = \mu^*(A) - \mu^*\Big( A \cap [E_1\cup E_2]^c \Big) \\
                \implies \mu^*(A) &\ge \mu^*\Big(A\cap(E_1\cup E_2)\Big) + \mu^*\Big( A \cap [E_1\cup E_2]^c \Big) \\
                \implies & E_1 \cap E_2 \in \mathcal{M}
            \end{align*}
        \end{itemize}
    \end{subproof}

    \begin{subproof}{\newline Claim 2 : $\mathcal{M}$ is a $\sigma$-algebra}
        Given $\{E_n\}_{n=1}^\infty$ be a countable collection of Lebesgue measurable sets and let $E=\bigcup_{n=1}^\infty E_n$. By proposition \ref{prop:disjoint_union_in_algebra}, there exists another countable \textbf{disjoint} collection of Lebesgue measurable sets $\{F_n\}_{n=1}^\infty$ such that $\bigcup_{n=1}^\infty F_n = \bigcup_{n=1}^\infty E_n=E$. 
        
        \noindent \newline For any integer $N\ge1$, we have $\bigcup_{n=1}^N F_n$ is Lebesgue measurable because $\mathcal{M}$ is an algebra. Hence, we have:
    
        \begin{align*}
            \mu^*(A) &= \mu^*\Bigg( A \cap \Bigg[ \bigcup_{n=1}^N F_n \Bigg] \Bigg) + \mu^*\Bigg( A \cap \Bigg[ \bigcup_{n=1}^N F_n \Bigg]^c \Bigg) \\
            &\ge \mu^*\Bigg( A \cap \Bigg[ \bigcup_{n=1}^N F_n \Bigg] \Bigg) + \mu^*\Bigg( A \cap \Bigg[ \bigcup_{n=1}^\infty F_n \Bigg]^c \Bigg) = \mu^*\Bigg( A \cap \Bigg[ \bigcup_{n=1}^N F_n \Bigg] \Bigg) + \mu^*(A\cap E^c)
        \end{align*}

        \noindent\newline By proposition \ref{prop:intersection_with_measurable_collection}, we have:
        \begin{align*}
            \mu^*(A) &\ge \sum_{n=1}^N \mu^*(A\cap F_n) + \mu^*(A\cap E^c)
        \end{align*}

        \noindent\newline Taking $N\to\infty$, we have:
        \begin{align*}
            \mu^*(A) &\ge \sum_{n=1}^\infty \mu^*(A\cap F_n) + \mu^*(A\cap E^c) \\
            &= \mu^*\Bigg( A \cap \Bigg[ \bigcup_{n=1}^\infty F_n \Bigg]\Bigg) + \mu^*(A\cap E^c) = \mu^*(A\cap E) + \mu^*(A\cap E^c)
        \end{align*}

        \noindent\newline Hence, $\mathcal{M}$ is closed under countable union and is a $\sigma$-algebra.
    \end{subproof}
\end{proof*}

\begin{proposition}{Translation invariance of Lebesgue measure}{trans_inv_of_lebesgue_measure}
    The Lebesgue (outer) measure is translation invariant.
\end{proposition}

\begin{proof*}[Proposition \ref{prop:trans_inv_of_lebesgue_measure}]
    Let $\mu^*:\mathcal{P}(\mathbb{R}) \to [0,\infty]$ be the outer measure. We have to prove that for every $E \in \mathcal{M}$, we have $\mu*(E)=\mu^*(E+x)$.

    \begin{subproof}{\newline Claim 1 : $\mu^*(E) \ge \mu^*(E+x)$}
        Let $\{I_n\}_{n=1}^\infty$ be the collection of open intervals that covers $E$. Then, $\{I_n+x\}_{n=1}^\infty$ covers $E+x$. Hence, we have:
        \begin{align*}
            \mu^*(E+x) &\le \mu^*\Bigg( \bigcup_{n=1}^\infty (I_n+x) \Bigg) \\
            &\le \sum_{n=1}^\infty \mu^*(I_n+x) \\
            &= \sum_{n=1}^\infty \mu^*(I_n) = \mu^*(E)
        \end{align*}
    \end{subproof}

    \begin{subproof}{\newline Claim 2 : $\mu^*(E) \le \mu^*(E+x)$}
        Let $\{I_n\}_{n=1}^\infty$ be the collection of open intervals that covers $E+x$. $\{I_n-x\}_{n=1}^\infty$ covers $E$. Hence, we have:
        \begin{align*}
            \mu^*(E) &\le \mu^*\Bigg( \bigcup_{n=1}^\infty (I_n-x) \Bigg) \\
            &\le \sum_{n=1}^\infty \mu^*(I_n-x) \\
            &= \sum_{n=1}^\infty \mu^*(I_n) = \mu^*(E+x)
        \end{align*}
    \end{subproof}

    \noindent From \textbf{Claim 1} and \textbf{Claim 2}, we have $\mu^*(E) = \mu^*(E+x) \ \ \forall E\in\mathcal{M}, x\in\mathbb{R}$. Hence, the Lebesgue (outer) measure is translation invariant.
\end{proof*}

\subsubsection{Borel Measure}
\begin{definition}[Borel measure]
    Let $E$ be a topological space and $\borel[E]$ be the Borel-$\sigma$-algebra generated from the open sets of $E$. Then, any measure defined on $(E, \borel[E])$ is called a \textbf{Borel measure}.
\end{definition}

\begin{proposition}{Non-Borel Lebesgue-measurable sets}{non_borel_measurable_sets}
    We know that all open intervals are Lebesgue measurable. Hence, $\borel \subset \mathcal{M}$, this implies the existence of \textbf{Non-Borel Lebesgue measurable sets}.
\end{proposition}

\begin{proof*}[Proposition \ref{prop:non_borel_measurable_sets}]
    Define $\mathcal{C}$ as the \textbf{Cantor set} and $c: [0, 1] \to [0,1]$ be the \textbf{Cantor function}. We define the following function $f:[0,1]\to[0,2]$ as:
    \begin{align*}
        f(x) = c(x) + x
    \end{align*}

    \noindent The function $f$ is strictly increasing defined on the unit interval. Hence, it maps Borel sets to Borel sets (\cite{book:royden}, exercises 45-47, chapter 2). 

    \noindent\newline Note that $f(\mathcal{C})$ has positive measure. Therefore, we can always choose non-measurable subsets from $f(\mathcal{C})$.

    \noindent\newline Define a non-Borel-measurable subset $N\subset[0,2]$ such that $f^{-1}(N)\subset \mathcal{C}$. Since the Cantor set has zero measure, $f^{-1}(N)$ has zero measure and is Lebesgue measurable. 

    \noindent\newline However, $f^{-1}(N)$ is not Borel measurable because then $f(f^{-1}(N))=N$ has to be Borel measurable, which is not true. Therefore \textbf{$f^{-1}(N)$ is Lebesgue measurable but not Borel measurable}. 
\end{proof*}


\noindent In the following section about the \textbf{Caratheodory Extension Theorem}, we will use it to prove the following results about Borel measures (Corollary \ref{coro:unique_borel_measure}):
\begin{itemize}
    \item There exists a unique Borel measure $\mu$ on $\mathbb{R}$ such that $\mu([a, b]) = b-a$.
    \item That unique Borel measure is the Lebesgue measure.
\end{itemize}